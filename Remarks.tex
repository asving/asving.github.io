\subsection{Renyi Entropy vs Shannon Entropy}

    Let $\mu = \sum_{i = 1}^{\ell} p_i \delta_{g_i}$ and assume that the $g_i$ generate a free group. In this case recall that $$\mathrm{h}_{RW}(\mu) = H(\mu) = -\sum_i p_i \log p_i.$$ Also 

    $$H_2(\mu) = -\log ||\mu||_2 = -\log \left( \sum_{i=1}^n p_i^2 \right).$$

    Assume $\mathrm{supp}(\mu)$ generates a free group, then $$\dim_2(\nu) \leq \frac{H_2(\mu)}{2\chi}.$$

    Let $\mu_n = \frac{1}{n} \sum_{i = 1}^n \delta_{g_i}$. Then $$H(\mu) = \log n$$ and $$H_2(\mu) = \log n.$$

    
    \subsection{Difficulties L1 versus L2 type approximations}

    One difficulty with combining the two approaches is that Sam's approach gives $L^1$ type bounds but to apply the methods based on spectral stuff we need $L^2$ type bounds. Amongst other things at scale $r$ Sam's method has a ``bad'' set of size $(\log r^{-1})^{-C}$ with $C$ depending on a variety of things ($H$, $\chi$, and the rate of separation amongst others). This is difficult because in the worst case this gives nowhere near the required control on $\chi_n$. Indeed we would get a bound something like $$||T^{\log n} \chi_n|| \leq (\log n)^{-C}.$$ One way to get around this problem is to do something like the following.

    \subsection{Difference between Furstenberg measures and self-similar sets}

    Just for clarity, we quickly remark why the Bourgain method to establish continuity does not work for self similar sets. Indeed let $\Phi = \{ \varphi_i \}_{1 \leq i \leq \ell}$ be a finite family of linear contractions on $\R$, i.e. maps of the form $\varphi_i(x) = r_i x + a_i$ with $|r_i| < 1$, and let $(p_1, \ldots , p_{\ell})$ be a probability vector. Then there is a unique Borel probability measure satisfying $$\mu = \sum_{i = 1}^{\ell} p_i (\varphi_{i})_{*} \mu,$$ where $(\varphi_{i})_{*} \mu = \mu \circ \varphi_i^{-1}$ is the push-forward. 

    Consider the operator $$T: L^2(\R) \to L^2(\R), \quad\quad f \mapsto \sum_{i = 1}^{\ell} p_i \cdot (f \circ \varphi_i^{-1}).$$ 

    It is straightforward to check that the $L^2$-adjoint of the operator $T$ is given as $$T^{*} f = \sum_{i = 1}^{\ell} p_i r_i \cdot  (f \circ \varphi_i).$$

    The big problem is now that we cannot argue that $1$ is in the discrete spectrum of $T$. This is where the proof breaks down. On the other hand, it should work if we have contractive bijections on the say torus. I don't think for self-similar sets it is sufficient to compactify $\R$ to $\mathbb{S}^1$ since it is not that clear what the metric is then going to be. 

    Another issue is that we will not in general have almost full dimension on the Lie group of similarities. For example if we let $\mu$ be a probability measure on the set of similarities of $\R$ generating the Bernoulli convolution with parameter $\lambda$ then $\mu^{*n}$ will be supported on the one dimensional sub-manifold of similarities with contraction ratio $\lambda^{n}$. In particular this means that we will have $$\mu^{*n}(B_r(x)) >> r$$ for $r$ at the sizes we care about and $\mu^{*n}$ a.e. $x$.

    \subsection{Effective Equidistribution on $G/\Gamma$ and the Furstenberg measure}

    Let $G = \mathrm{SL}_2(\R)$ and let $\Gamma < G$ be a cocompact lattice. Write $X = G/\Gamma$. Let $d$ be a right-invariant metric on $G$ and consider the metric $$d_X(x,y) = \inf_{\gamma \in \Gamma} d(g_x\gamma, g_y) $$ for $x = g_x\Gamma$ and $y = g_y\Gamma$ in $X$. It is well known that since $X$ is compact there is $c > 0$ such that for all $\eps > 0$ the map projection map $\pi : G \to X$, $g \mapsto g\Gamma$ is injective on $B_{\eps}(g)$. 

    In this section let $\mu$ be a probability measure on $G$ and assume that for all $f\in C^{\infty}(X)$,  $$\int f(ge) \, \mu^{*n}(g)  = \int f \, d\Haarof{X} + C\cdot \max (\mathrm{Lip}(f), ||f||_{\infty}) \cdot e^{-\theta n}$$ for $\theta$ and $C$ absolute constants depending on $\Gamma$ and $\mu$.

    Let $f  = 1_{B_{\delta}(x)}$ for some $\delta > 0$ smaller than the injectivity radius. 

    \subsection{A non-continuous map}

    Write $\mathscr{H} = L^2(\mathbb{S}^1)$. Consider the continuous linear operator $$E: B(\mathscr{H}) \to B(\mathscr{H}) $$ that sends $$\rho(g) \to \tau(g).$$ By definition it maps $$E(S_{\mu}) = T_{\mu}$$ for every probability measure $\mu$. Note that there is no reasonable topology on $B(\mathscr{H})$ such that $E$ is continuous. Indeed, since $||S_{\mu}|| < 1$ it holds that $S_{\mu^{*n}} \to 0$ in the operator norm. However $T_{\mu^{*n}} = E(S_{\mu^{*n}})$ has always operator norm $\geq 1$ since $T_{\mu^{*n}} 1 = 1$. Therefore $T_{\mu^{*n}}$ does not converge to zero and hence $E$ is not continuous. 

    If $E$ was continuous, then our life would be very easy as it would imply the existence of roots of $\tau(\lambda_{r,n})$. Below we keep some wrong arguments, just as a reminder.
    
    
    We want to take here the weak operator topology on $B(\mathscr{H})$ since we want that if $g_i \to g$ that to then $\rho(g_i) \to \rho(g)$. We also need some slight regularity condition on $\mu$ to ensure that $T_{\mu}$ is bounded. Note that $S_{\mu}$ is always bounded. Also it holds that $$ E(S_{\mu_1 * \mu_2}) = T_{\mu_1 * \mu_2} = T_{\mu_1} \circ T_{\mu_2} = E(S_{\mu_1}) \circ E(S_{\mu_2}) $$ so $E$ is a ring homomorphism on the space of sufficiently regular measures. 



Note that $E$ is only defined on the space $\mathscr{I} = \overline{\{ \sum_{i = 1}^k a_i S_{\mu_i} \}}$. For simplicity we should also restrict to symmetric measures.

Q: Is $S_{r,n} \in \mathscr{J}$ for say odd $n$? I think it is. This should follow from the methods from the continuous functional calculus, as exposed in the functional analysis book by Einsiedler and Ward.

Fix a measure $\mu$ and hence an operator $S_{\mu}$. Then by Lemma 12.38 it holds that $\sigma(S_{\mu^{*n}}) = \sigma(S_{\mu^{*n}})$ and $||S_{\mu^{*n}}||_{\mathrm{op}} = ||S_{\mu}||^n$. Also denote by $\C[z]$ the space of polynomials on $\C$. Then the map $$(\C[z], ||\cdot||_{\infty, \sigma(S_\mu)}) \to B(\mathscr{H}), \quad\quad p = \sum_{i= 0}^n a_i z^i \mapsto p(S_{\mu}) = \sum_{i = 0}^n a_n S_{\mu^{*n}}$$ is linear and continuous (indeed, an isometry). Thus it extends uniquely, using Stone-Weierstrass, to a map defined on $C(\sigma(S_{\mu})).$ Since for odd $n$ and since $S_{\mu}$ is self-adjoint, the map $x \to x^{1/n}$ is well-defined, it follows that $S_{r,n} \in \mathscr{J}$. 

Therefore we have constructed an analogue of the the operator $T_{r,n}$, namely the operator $E( S_{r,n})$. We note that by uniqueness $T_{r,n}^n = \tau(\lambda_{r,n})$. Assume for the moment that $\sigma_{\mathrm{ess}}(T_{r,n}) < 1$. Then (using that $T_{r,n} 1 = 1$) there is $\varphi \in L^2(\mathbb{S}^1)$ such that $T^{*}_{r,n} \varphi = \varphi$ and hence also $$\tau(\lambda_{r,n})^{*}\varphi = T_{r,n}^{n,*} \varphi = \varphi.$$ So the FM of $\lambda_{r,n}$ is in $L^2$ and $\varphi$ would be its density.

Say we can show that $S_{r,n} \to S_r$. Then we also have an operator $T_r$. $S_r$ and $T_r$ should be close in operator norm (maybe we should work with operator norm after all). So it would suffice to show that $||S_r \chi_{\ell}|| \leq \frac{1}{2}$ for $\ell$ large enough. Ok if everything converges in operator norm, I am quite sure that everything works. Let's check all that more preciesly. 

Let $\mathscr{M}_f^1(G)$ be the space of finitely supported probability measures on $G$. Then for each such measure we can make sense of $\rho(\mu)$ and $\tau(\mu)$.             