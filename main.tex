\documentclass[10pt, english,oneside,reqno]{amsart}
%Basics
\usepackage[utf8]{inputenc}
\usepackage[english]{babel}
\usepackage{hyperref}
\usepackage{color}
\usepackage{enumerate}
\usepackage{fancyhdr}

%General Math
\usepackage{amsmath}
\usepackage{amscd}
\usepackage{amsfonts}
\usepackage{amsthm}
\usepackage{amstext}
\usepackage{amssymb}
\usepackage{amsbsy}
\usepackage{mathrsfs}
\usepackage{tikz-cd}

%Graphics
\usepackage{float}
\usepackage{caption}
\usepackage{subcaption}
\usepackage{graphicx}

%Bibliography
\usepackage[alphabetic,abbrev]{amsrefs}
\def\bibfont{\footnotesize}
\usepackage{url}

%TheoremStyles
\theoremstyle{plain}
\newtheorem{theorem}{Theorem}[section]
\newtheorem{conjecture}[theorem]{Conjecture}
\newtheorem{definition}[theorem]{Definition}
\newtheorem{proposition}[theorem]{Proposition}
\newtheorem{corollary}[theorem]{Corollary}
\newtheorem{lemma}[theorem]{Lemma}
\newtheorem{example}[theorem]{Example}
\newtheorem{step}{Step}
\newtheorem*{claim}{Claim}
\theoremstyle{remark}
\newtheorem{rem}[theorem]{Remark}
\newtheorem{remarks}[theorem]{Remarks}
\newtheorem{remark}[theorem]{Remark}
\setcounter{tocdepth}{1}

%Counters
\numberwithin{equation}{section}
\numberwithin{figure}{section}


\newcommand{\usnote}[1]{\marginpar{\color{blue}\tiny [US] #1}}
\newcommand{\nsnote}[1]{\marginpar{\color{esperance}\tiny [NS] #1}}
\newcommand{\usadd}[1]{{\color{red}{\tiny [US]} #1}}

\newcommand{\G}{PSL_2(\R)}

\DeclareMathOperator{\var}{Var}
\DeclareMathOperator{\vart}{VAR}
\DeclareMathOperator{\adj}{Adj}
\DeclareMathOperator{\id}{Id}
\DeclareMathOperator{\Tr}{Tr}
\DeclareMathOperator{\im}{Im}
%GeneralNotation
\newcommand{\norm}[1]{|| #1 ||}
\newcommand{\abs}[1]{| #1 |}
\newcommand{\ceil}[1]{\left\lceil #1 \right\rceil}
\newcommand{\floor}[1]{\left\lfloor #1 \right\rfloor}
\newcommand{\eps}{\varepsilon}
\newcommand{\tr}{\mathrm{tr}}
\newcommand{\re}[1]{\mathrm{Re}(#1)}
\newcommand{\im}[1]{\mathrm{Im}(#1)}
\newcommand{\Id}{\mathrm{Id}}

%Numbers
\newcommand{\N}{\mathbb{N}}
\newcommand{\Z}{\mathbb{Z}}
\newcommand{\Q}{\mathbb{Q}}
\newcommand{\R}{\mathbb{R}}
\newcommand{\E}{\mathbb{E}}
\newcommand{\C}{\mathbb{C}}
\newcommand{\F}{\mathbb{F}}
\newcommand{\A}{\mathbb{A}}
\newcommand{\D}{\mathbb{D}}
\newcommand{\Alg}{\overline{\mathbb{Q}}}


%Lie Groups
\newcommand{\SL}{\mathrm{SL}}
\newcommand{\GL}{\mathrm{GL}}
\newcommand{\PSL}{\mathrm{PSL}}
\newcommand{\PGL}{\mathrm{PGL}}
\newcommand{\SO}{\mathrm{SO}}
\newcommand{\SU}{\mathrm{SU}}
\newcommand{\PSO}{\mathrm{PSO}}
\newcommand{\PSU}{\mathrm{PSU}}
\newcommand{\Ad}{\mathrm{Ad}}
\newcommand{\ad}{\mathrm{ad}}

%Spaces
\newcommand{\Banach}{\mathscr B}
\newcommand{\OpBanach}{\mathrm{Op}(\mathscr B)}
\newcommand{\Hilbert}{\mathscr H}
\newcommand{\Schwartz}[1]{\mathscr{S}(#1)}
\newcommand{\SchwartzWeylGroup}[1]{\mathscr{S}_{W_G}(#1)}
\newcommand{\DifferentialOps}[1]{\mathscr{D}(#1)}
\newcommand{\Operator}{L}
\newcommand{\UnitaryOperator}{U}
\newcommand{\Spectrum}[1]{\sigma(#1)}
\newcommand{\SpectralRadius}[1]{\rho(#1)}
\newcommand{\EssentialSpectrum}[1]{\sigma_e(#1)}
\newcommand{\EssentialSpectralRadius}[1]{\rho_e(#1)}
\newcommand{\DiscreteSpectrum}[1]{\sigma_{\mathrm{disc}}(#1)}
\newcommand{\FreeGroup}[1]{\mathbb{F}_{#1}}

%Measures
\newcommand{\Haarof}[1]{m_{#1}}
\newcommand{\LimitMeasure}[1]{m_{#1}}
\newcommand{\StationaryMeasure}[1]{\mathrm{m}_{#1}}
\newcommand{\supp}[1]{\mathrm{supp}(#1)}
\newcommand{\SphericalPlancharelMeasure}{\nu_{\mathrm{sph}}}
\newcommand{\FurstenbergMeasure}{\nu_{\mathrm{F}}}
\newcommand{\FurstenbergDensity}{\psi_{\mathrm{F}}}

%SpectralGap
\newcommand{\spectralgap}[2]{\mathrm{gap}(#1,#2)}
\newcommand{\KazConstRep}[3]{\mathrm{Kaz}(#1,#2,#3)}
\newcommand{\KazConst}[2]{\mathrm{Kaz}(#1,#2)}

%CayleyGraphs
\newcommand{\Cay}{\mathrm{Cay}}
\newcommand{\Schreier}{\mathrm{Sch}}

%RepresentationTheorey
\newcommand{\LeftRegRep}[1]{\lambda_{#1}}
\newcommand{\PrincialSeries}[1]{\rho_{#1}}
\newcommand{\SpectrumPrincipalSeriesZero}[1]{\sigma_{#1}}
\newcommand{\PrincialSeriesProjectionMax}[1]{P_{#1}}
\newcommand{\HyperbolicTransfromMeasure}{\nu_{\mathrm{hyp}}}

%AdditiveCombinatorics
\newcommand{\DoubleConst}[1]{\mathrm{\sigma}(#1)}
\newcommand{\DiffConst}[1]{\mathrm{\delta}(#1)}
\newcommand{\AddEnergy}[2]{E(#1,#2)}
\newcommand{\MetEnt}[2]{\mathcal{N}_{#2}(#1)}



\title[AI and Mathematics]{AI and Mathematics\\[1em]
  {Asvin G, Constantin Kogler}\\[0.5em]
  {\textnormal{\lowercase{asving@ias.edu, kogler@ias.edu}}}\\[0.5em]
  { \today}
}

%\title{
%  How to build an AI mathematician?\\[0.5em]e
%  Constantin Kogler \\ 
%  {\normalfont kogler@ias.edu} \\
%  \today
%}


\begin{document}

    \begin{abstract}
        In this essay, we present our personal views on the future devolvement of mathematics at the dawn of revolutionary reasoning machines. We will promptly discuss the current state of affairs, make predictions, and suggest what might be necessary to improve mathematical abilities of current reasoning machines.
    \end{abstract}


    \maketitle




    \section{Introduction}

     According to Sigmund Freud, \textit{The Three Insults on Humanity}, or in the original German \textit{Die drei Kränkungen der Menschheit}, are major discoveries showing that we humans are less unique and mighty than we may hope. First, the cosmological insult: the earth revolves around the sun and we are not the center of the universe. Second, Darwin brought along the biological insult: We stem from apes and are not a special species on planet earth. Finally, Freud delivered the third insult: We are not in full control of ourselves, yet our subconsciousness guides us strongly.

    The fourth and final insult appears just to be around the corner: Machines are better at thinking than we are. It appears obvious that this ultimate insult will have happened within the next 50 years, but has not happened yet. 
    
    In this short essay, I want to clarify my personal thoughts concerning this crucial topic and discuss the following predictions:

    \begin{enumerate}
        \item For machines to become superhuman at mathematics, significant new ideas are required and just scaling the current techniques is not sufficient.
        \item The mathematician's profession will drastically change. Creating definitions, building theories and mathematical exposition will become significantly more important.
        \item We will first have autonomous robots before an AI will solve the Riemann hypothesis.
        \item The Riemann Hypothesis will not be solved by an AI autonomously in the next 10 years.
        \item Some of the open problems discussed in my PhD thesis will still be open in 10 years.
        \item My PhD thesis will not be fully formalised in the next 2 years.
        \item The first significant open problem an AI will solve will be in combinatorics.
    \end{enumerate}

    These predictions are in decreasing order of importance and are of course interconnected. In particular, if I am wrong about (1), I might also be wrong about (4), (5) and (6). 

    I also want to stress that there is nothing exceptional about my PhD thesis. It is just a nieche piece of mathematical writing I am familiar with and I have a sense of the difficulty of the problems involved. Also, I am almost defintely the only person on planet earth who has fully read every page of this very minor contribution to human knowledge. So my predictions (4) and (5) should rather be understood as addressing the dissemination of AI on nieche mathematical topics. 

    Indeed, imagine a scenario in which a powerful company is willing to spend 10 billion dollars on computational resources to prove the Riemann Hypothesis, yet won't use the corresponding at least 10 million dollars to address the open problems explored in my PhD thesis, or in any of the thousand or so other mathematics dissertations completed this year.
    


    \section{The current state of AI for mathematics}


    \subsection{Informal Reasoning of LLMs} The leading large language models are becoming impressive at mathematical reasoning. They were trained on the entire corpus of human mathematical knowledge and can insert what they have read in the right context. They are capable of providing correct proofs of basic mathematical statements, also of ones that have not appeared in print in the given precise form.  When 

    What they currently lack is an in depth understanding of frontier mathematics and the ability of coming up with new ideas. Also, they hallucinate about what actually has been proven already, while not being able to provide easy arguments for minor questions written nowhere in the literature.

    \subsection{Formal Reasoning} 

    \subsection{Autoformalisation}
    
    \section{How to improve reasoning machines}   

    \section{My predictions}

 
\end{document}
